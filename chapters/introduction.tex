\cleardoublepage
\phantomsection
\addcontentsline{toc}{chapter}{Введение}
\chapter*{Введение}
\label{chap:introduction}

Это --- пример оформления выпускной квалификационной работы в \LaTeX.
Набирая текст в этой системе, не нужно задумываться об оформлении документа по ГОСТу.
Требуется только разметить текст согласно структуре работы, и \LaTeX{} автоматически соберёт документ, соответствующий ГОСТу.

Обычный текст вводится так, как есть.
Отдельные предложения имеет смысл отделять друг от друга новыми строками: это помогает в редактировании, но не обчзательно.
В итоговом документе такие одиночные переводы строк не учитываются.

Абзацы отделяются друг от одной или более пустыми строками.
Каждый абзац должен содержать единственную, законченную мысль.

Команды, начинающиеся с символа \verb|\|, являются специальными инструкциями \TeX{} или \LaTeX.
Чтобы оставить пробел после команды, нужно поставить фигурные скобки: \verb|\TeX{}|.

Помимо комманд, также существуют окружения, использующиеся для больших блоков содержимого:
\begin{verbatim}
\begin{environment}
content
\end{environment}
\end{verbatim}

Внешние кавычки печатаются <<так>>.
Внутренние кавычки ставятся ,,так``.
Также можно использовать команду \verb|\enquote| \enquote{для автоматического \enquote{определения} уровня вложенности}.

Есть 3 основных типа тире.
Дефис \verb|-| используется для разделения сложных слов и некоторых предлогов: квадратно-гнездовой
Короткое тире \verb|--| \en{(en-dash)} встречается в числовых диапазонах: 1970--2015.
Длинное тире \verb|---| \en{(em-dash)} --- это обычное тире в предложениях.

Для набора английского текста используйте команду \verb|\en| или окружение \verb|english|.

Боле подробно процесс набора текста описан в Интернете или в статье~\cite{oetiker1995not}.
